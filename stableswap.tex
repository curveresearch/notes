% Options for packages loaded elsewhere
\PassOptionsToPackage{unicode}{hyperref}
\PassOptionsToPackage{hyphens}{url}
\PassOptionsToPackage{dvipsnames,svgnames,x11names}{xcolor}
%
\documentclass[
]{article}
\usepackage{amsmath,amssymb}
\usepackage{iftex}
\ifPDFTeX
  \usepackage[T1]{fontenc}
  \usepackage[utf8]{inputenc}
  \usepackage{textcomp} % provide euro and other symbols
\else % if luatex or xetex
  \usepackage{unicode-math} % this also loads fontspec
  \defaultfontfeatures{Scale=MatchLowercase}
  \defaultfontfeatures[\rmfamily]{Ligatures=TeX,Scale=1}
\fi
\usepackage{lmodern}
\ifPDFTeX\else
  % xetex/luatex font selection
\fi
% Use upquote if available, for straight quotes in verbatim environments
\IfFileExists{upquote.sty}{\usepackage{upquote}}{}
\IfFileExists{microtype.sty}{% use microtype if available
  \usepackage[]{microtype}
  \UseMicrotypeSet[protrusion]{basicmath} % disable protrusion for tt fonts
}{}
\makeatletter
\@ifundefined{KOMAClassName}{% if non-KOMA class
  \IfFileExists{parskip.sty}{%
    \usepackage{parskip}
  }{% else
    \setlength{\parindent}{0pt}
    \setlength{\parskip}{6pt plus 2pt minus 1pt}}
}{% if KOMA class
  \KOMAoptions{parskip=half}}
\makeatother
\usepackage{xcolor}
\usepackage[margin=1.35in]{geometry}
\usepackage{listings}
\newcommand{\passthrough}[1]{#1}
\lstset{defaultdialect=[5.3]Lua}
\lstset{defaultdialect=[x86masm]Assembler}
\setlength{\emergencystretch}{3em} % prevent overfull lines
\providecommand{\tightlist}{%
  \setlength{\itemsep}{0pt}\setlength{\parskip}{0pt}}
\setcounter{secnumdepth}{-\maxdimen} % remove section numbering
\newlength{\cslhangindent}
\setlength{\cslhangindent}{1.5em}
\newlength{\csllabelwidth}
\setlength{\csllabelwidth}{3em}
\newlength{\cslentryspacingunit} % times entry-spacing
\setlength{\cslentryspacingunit}{\parskip}
\newenvironment{CSLReferences}[2] % #1 hanging-ident, #2 entry spacing
 {% don't indent paragraphs
  \setlength{\parindent}{0pt}
  % turn on hanging indent if param 1 is 1
  \ifodd #1
  \let\oldpar\par
  \def\par{\hangindent=\cslhangindent\oldpar}
  \fi
  % set entry spacing
  \setlength{\parskip}{#2\cslentryspacingunit}
 }%
 {}
\usepackage{calc}
\newcommand{\CSLBlock}[1]{#1\hfill\break}
\newcommand{\CSLLeftMargin}[1]{\parbox[t]{\csllabelwidth}{#1}}
\newcommand{\CSLRightInline}[1]{\parbox[t]{\linewidth - \csllabelwidth}{#1}\break}
\newcommand{\CSLIndent}[1]{\hspace{\cslhangindent}#1}
\usepackage{draftwatermark}
\SetWatermarkLightness{0.95}
\usepackage[symbol]{footmisc}
\renewcommand{\thefootnote}{\fnsymbol{footnote}}
\usepackage{authblk}
\author{Chanho Suh\\ {\small chanho@curveresearch.org}}
\affil{Curve Research\footnote{Curve Research is a community organization funded through the Curve DAO grants program and is not affiliated with Curve Finance (Swiss Stake GmbH).  Neither Curve Research nor Curve DAO are responsible for any damages that result from use of the provided information or guarantee its accuracy.}}

\usepackage{xcolor}
\usepackage{pifont}

\lstset{
  language=Python,
  numbers=left,
  numberstyle=\tiny,
  stepnumber=5,
  frame=lines,
  backgroundcolor=\color{yellow!25},
  columns=fixed,
  escapeinside={/*@}{@*/},
  basicstyle=\ttfamily\scriptsize,
  breaklines=true,
  commentstyle=\color{gray},
  keywordstyle=\color{blue},
  stringstyle=\color{green},
  captionpos=b,
  otherkeywords={}
}

\makeatletter
\AtBeginDocument{%
  % Counter `lstlisting' is not defined before `\begin{document}'
  \newcounter{llabel}[lstlisting]%
  \renewcommand*{\thellabel}{%
    \ifnum\value{llabel}<0 %
      \@ctrerr
    \else
      \ifnum\value{llabel}>10 %
        \@ctrerr
      \else
        \protect\footnotesize\ding{\the\numexpr\value{llabel}+201\relax}%
      \fi
    \fi
  }%
}
\newlength{\llabelsep}
\setlength{\llabelsep}{5pt}

\newcommand*{\llabel}[1]{%
  \begingroup
    \refstepcounter{llabel}%
    \label{#1}%
    \llap{%
      \thellabel\kern\llabelsep
      \hphantom{\lst@numberstyle\the\lst@lineno}%
      \kern\lst@numbersep
    }%
  \endgroup
}
\makeatother
\ifLuaTeX
  \usepackage{selnolig}  % disable illegal ligatures
\fi
\IfFileExists{bookmark.sty}{\usepackage{bookmark}}{\usepackage{hyperref}}
\IfFileExists{xurl.sty}{\usepackage{xurl}}{} % add URL line breaks if available
\urlstyle{same}
\hypersetup{
  pdftitle={Curve Stableswap: From Whitepaper to Vyper},
  colorlinks=true,
  linkcolor={Maroon},
  filecolor={Maroon},
  citecolor={Blue},
  urlcolor={Blue},
  pdfcreator={LaTeX via pandoc}}

\title{Curve Stableswap: From Whitepaper to Vyper}
\usepackage{etoolbox}
\makeatletter
\providecommand{\subtitle}[1]{% add subtitle to \maketitle
  \apptocmd{\@title}{\par {\large #1 \par}}{}{}
}
\makeatother
\subtitle{v0.2 (draft version)}
\author{true}
\date{\today}

\begin{document}
\maketitle

\hypertarget{introduction}{%
\section{Introduction}\label{introduction}}

The stableswap invariant was derived by Michael Egorov and promulgated
in the \href{https://curve.fi/files/stableswap-paper.pdf}{stableswap
whitepaper} (Egorov 2019). The whitepaper clearly explained the
invariant and its implications for DeFi; however, there are differences
with how it is implemented in practice, currently across hundreds of
live contracts across Ethereum and other layer 2s and chains.

Particularly important details for the practitioner but not given in the
whitepaper are:

\begin{enumerate}
\def\labelenumi{\arabic{enumi}.}
\tightlist
\item
  implementation of fees, both for exchanges and adding liquidity
\item
  practical solution procedures for the invariant and related quantities
  in integer arithmetic
\end{enumerate}

The practitioner seeking to understand the live functionality of the
stableswap pools must look toward the vyper code for help, which while
very readable, has minimal and sometimes outdated comments and
explanation. To understand the vyper code, the reader must have a solid
grasp of the whitepaper in order to translate to the appropriate
variables and understand various tweaks needed for implementation.

This note seeks to close the gap between the whitepaper and the vyper
contracts. It seeks to give a consistent derivation of the necessary
mathematics, using the notation and language of the contracts. At the
same time, it points out and explains the ``grungy'' changes to
calculations needed to ensure secure and safe operation on the Ethereum
Virtual Machine.

\hypertarget{preliminaries-notation-and-conventions}{%
\section{Preliminaries (notation and
conventions)}\label{preliminaries-notation-and-conventions}}

\hypertarget{stableswap-equation}{%
\subsection{Stableswap equation}\label{stableswap-equation}}

This is the original stableswap equation:
\[ A \cdot n^n  \sum_i x_i + D = A \cdot n^n \cdot D + \frac{D^{n+1}}{n^n \prod_i x_i}\]

In the vyper code, the amplification constant \(A\) actually refers to
\(A \cdot n^{n-1}\), so the equation becomes:

\[ A \cdot n  \sum_i x_i + D = A \cdot n \cdot D + \frac{D^{n+1}}{n^n \prod_i x_i}\]

This is the form we use for all our derivations.

\hypertarget{coin-balances}{%
\subsection{Coin balances}\label{coin-balances}}

We denote the coin balances (as in the contract code) with \(x_i\),
\(x_j\) etc. In the context of a swap, \(i\) is the ``in'' index and
\(j\) is the ``out'' index.

Balances are in special units. They are \emph{not} native token units.
For example, if \(x_i\) represents the USDC amount, one whole token
amount would not be \(1000000\) as might be assumed from the 6 decimals
for USDC. Instead \(x_i\) would be \(1000000000000000000\) (18 zeroes).
All the \(x\) balances should be assumed to be in the same units as
\(D\). For lack of a better term, sometimes we will call the \(x\)
balances \emph{virtual balances}, as the amount of \(D\) per LP token is
often referred to as ``virtual price''. In the vyper code, virtual
balances are notated by \(xp\) while \(x\) is in native token units.

While putting balances into virtual units often involves only a change
of decimals, this is not always the case and it is helpful to think of
this more generally, particularly for metapools and thinking about the
cryptoswap invariant. The stableswap equation assumes a 1:1 peg between
coins. This means the balances being used must reflect the proper value
in the common units of D being used. For the example of USDC, this means
adjusting simply the decimals. For a rebasing token however, it may not
be. Indeed, for metapools, when exchanging the basepool LP token for the
main stable, the basepool LP token conversion into D units must take
into account the accrued yield. This is done by multiplying the LP token
amount by the basepool virtual price.

\hypertarget{solving-for-d}{%
\section{\texorpdfstring{Solving for
\(D\)}{Solving for D}}\label{solving-for-d}}

Since the arithmetic mean is greater than the geometric mean (unless the
balances \(x_i\) are equal, in which case the means are identical), the
form of the equation suggests that there ought to be a \(D\) in-between
the means that satisfies the equation.

To see this rigorously, we use the auxiliary function:

\[ f(D) = A \cdot n \cdot (D - \sum_i x_i)  + D \cdot (\frac{D^{n}}{n^n \prod_i x_i} - 1) \]

Let \(P = n\ (\prod_i x_i)^{\frac{1}{n}}\) and \(S = \sum_i x_i\). This
is a continuous function (away from zero balances) with \(f(P) < 0\) and
\(f(S) > 0\). So there is a D such that \(P < D < S\) and \(f(D) = 0\).
In particular, note

\[ f'(D) = A\cdot n + (n+1) \frac{D^n}{n^n \prod_i x_i} - 1 \]

the derivative of \(f\), is positive (assuming \(A >= 1\)), so \(f\) is
strictly increasing and there is a unique \(D\) that solves
\(f(D) = 0\).

\hypertarget{newtons-method}{%
\subsection{Newton's method}\label{newtons-method}}

The stableswap contracts utilize Newton's method to solve for \(D\). It
is easy to check \(f'' > 0\), i.e.~\(f\) is convex. An elementary
argument later will show that this guarantees convergence of Newton's
method starting with initial point \(S\) to the solution.

The vyper code (Curve Finance 2020) is:

\begin{lstlisting}[language=Python, numbers=left, firstnumber=193, caption={Calculation of D, the stableswap invariant}, label=get_D]
@pure             /*@\llabel{pure}@*/
@internal
def get_D(xp: uint256[N_COINS], amp: uint256) -> uint256:
    S: uint256 = 0 
    for _x in xp:
        S += _x
    if S == 0:
        return 0

    Dprev: uint256 = 0
    D: uint256 = S            /*@\llabel{initial}@*/
    Ann: uint256 = amp * N_COINS
    for _i in range(255):
        D_P: uint256 = D
        for _x in xp:
            D_P = D_P * D / (_x * N_COINS)            /*@\llabel{zero_division}@*/
        Dprev = D
        D = (Ann * S + D_P * N_COINS) * D / ((Ann - 1) * D + (N_COINS + 1) * D_P)            /*@\llabel{formula}@*/
        # Equality with the precision of 1           /*@\llabel{convergence}@*/
        if D > Dprev:
            if D - Dprev <= 1:
                break
        else:
            if Dprev - D <= 1:
                break
    return D            /*@\llabel{revert}@*/
\end{lstlisting}

This code is used with minimal difference between all the stableswap
contracts.

\ref{pure} In general, pure functions for calculations are to be
preferred as they simplify logic and testing, while saving gas costs, as
reading storage is among the most expensive EVM operations.

\ref{initial} The initial guess for Newton's Method is the sum of all
the balances. From our previous remarks, we know this is the maximum
possible value for \(D\) and as we iterate, our guesses will
decreasingly converge to the solution.

\ref{zero_division} If we passed in a zero balance, this will revert
from a division by zero. Note naively computing \(D_p\) by multiplying
out the numerator of \(\frac{D^{n+1}}{n^n \prod_i x_i}\) easily
overflows \(2^{256} (\approx 10^{77})\) for reasonable balances
(remember balances are normalized to 18 decimals). Iteratively
truncating by integer division in this manner is a reasonable tradeoff.

\ref{formula} The iterative formula is easily derived: \[\begin{aligned}
d_{k+1} &= d_k - \frac{f(d_k)}{f'(d_k)} \\
&= d_k - \frac{A n (d_k - \sum_i x_i)  + d_k(\frac{d_k^{n}}{n^n \prod_i x_i} - 1)}{\frac{(n+1)d_k^n}{n^n \prod_i x_i} + An - 1} \\
&= \frac{An\sum_i x_i + \frac{nd_k^{n+1}}{n^n \prod_i x_i}}{\frac{(n+1)d_k^n}{n^n\prod_i x_i} + An - 1} \\
&= \frac{AnS + nD_p(d_k)}{(n+1)\frac{D_p(d_k)}{d_k} + An - 1} \\
&= \frac{(AnS + nD_p(d_k))d_k}{(n+1)D_p(d_k) + (An-1)d_k}\\
\end{aligned}\]

where \(S = \sum_i x_i\) and
\(D_p(d_k) = \frac{d_k^{n+1}}{n^n \prod_i x_i}\).

Note the careful formulation to avoid loss of precision by integer
division. The quantities \(D\), \(D_p\), \(S\), and \(d_k\) all are of
magnitude \(\operatorname{TVL} \cdot 10^{18}\), where
\(\operatorname{TVL}\) is the order of magnitude of the pool value in
dollars. This means the numerator is of magnitude
\(An\cdot (\operatorname{TVL}\cdot 10^{18})^2\) while the denominator
has magnitude \(An\cdot \operatorname{TVL}\cdot 10^{18}\). This means we
can expect the result to be of magnitude
\(\operatorname{TVL}\cdot 10^{18}\) and to be accurate within 1 smallest
unit of \(D\).

\ref{convergence} Convergence is gone into more detail in the next
section. For now, note we can achieve extremely precise convergence due
to the convexity of the curve we move along.

\ref{revert} For safety, later versions choose to revert if the 255
iterations are exhausted before converging. Note that typically
conditions 4-6 iterations are expected, with a handful more under
extreme imbalances. Stableswap math has proven capable in both numerical
and live testing of handling imbalances that arise in practice with
significantly less two dozen iterations.

\hypertarget{rate-of-convergence}{%
\subsubsection{Rate of convergence}\label{rate-of-convergence}}

Convergence follows from convexity of \(f\). However we need much better
than that, we need to reduce the distance to the solution by half each
time, otherwise 255 iterations is not sufficient. Also, in practice,
exceeding more than half a dozen iterations is not sufficiently gas
efficient enough to be competitive. We will in fact demonstrate
quadratic convergence, which means the Netwon estimate will double in
accuracy on each iteration.

First, to see convergence is straightforward. First recall that
\(f(P) < 0\) and \(f(S) > 0\) and that there is exactly one zero in
\([P, S]\) for \(f\).

We have the first derivative:

\[ f'(D) = A\cdot n + (n+1) \frac{D^n}{n^n \prod_i x_i} - 1 \]

and the second derivative:

\[ f''(D) = n (n+1) \frac{D^{n-1}}{n^n \prod_i x_i} \]

With \(A >= 1\), \(f' > 0\) and \(f'' > 0\) everywhere, and in
particular on \([P, S]\) the domain of interest.

The formula for Newton's method is:

\[ d_{k+1} := d_k - \frac{f(d_k)}{f'(d_k)} \]

Since \(f\) is convex, its graph lies above every tangent line. In
particular, supposing our initial guess \(S\) is not the solution, for
every iteration of Newton's method, \(f(d_{k+1}) > 0\) since the tangent
line approximation intersects the \(x\)-axis at \(d_{k+1}\). Since
\(f'(d_k) > 0\) also, we see that \(d_{k+a}\) is always to the left of
\(d_k\). The solution \(D\) we are seeking is always to the left of any
iterate (because \(f(D) = 0\) while \(f(d_k) > 0\)) so the sequence
\(d_k\) converges to \(c\). We claim \(f(c) = 0\) and thus \(c = D\).
This can be seen from the iterative formula. Since \(d_{k} - d_{k+1}\)
get arbitrarily small, \(\frac{f(d_k)}{f'(d_k)}\) gets arbitrarily
small. But the denominator \(f'(d_k)\) has a max on \([c, S]\) so the
numerator must be getting arbitrarily small,
i.e.~\(f(d_k) \rightarrow 0\), which implies \(f(c) = 0\).

For quadratic convergence, we first need to derive an inequality using a
couple applications of the mean-value theorem.

Let \(\delta_k = d_k - c\). Then
\(f'(\eta_k) (\delta_k) = f(d_k) - f(c) = f(d_k)\) for
\(c\leq \eta_k \leq d_k\). Rewriting,
\(\delta_k = \frac{f(d_k)}{f'(\eta_k)}\).

Then the Newton iteration can be rewritten as:

\[\begin{aligned}
\delta_{k+1} &:= \delta_k - \frac{f(d_k)}{f'(d_k)} \\
&= \frac{f(d_k)}{f'(\eta_k)} - \frac{f(d_k)}{f'(d_k)} \\
&= \frac{ f(d_k) (f'(d_k) - f'(\eta_k)) }{f'(\eta_k) f'(d_k)} \\
&= \frac{ f(d_k) f''(\xi_k)(d_k - \eta_k)}{f'(\eta_k) f'(d_k)} \\
&= \frac{f''(\xi_k)}{f'(d_k)} (d_k - \eta_k) \delta_k\\
&\leq \frac{f''(\xi_k)}{f'(d_k)} \delta_k^2\\
&\leq \frac{f''(d_k)}{f'(d_k)} \delta_k^2\\
&\leq \frac{n (n+1) d_k^{n-1} }{(n+1) d_k^n + (An - 1) n^n \prod_i x_i  }\delta_k  \cdot \delta_k\\
&\leq \frac{n (n+1) d_k^{n-1} }{(n+1) d_k^n }\delta_k  \cdot \delta_k\\
&\leq n\frac{\delta_k}{d_k } \cdot \delta_k\\
&\leq n\frac{\delta_k}{c } \cdot \delta_k\\
\end{aligned}\]

Note that \(\frac{\delta_k}{c}\) is the distance of our estimate to the
solution, as a percentage of the solution.

In particular, if we can get \(n\frac{\delta_0}{c} < m\) for the first
estimate, then for subsequent estimates, we find:

\[ \delta_{k} = m^{2^k - 1} \delta_0\]

So if \(m\) is less than 1, say \(1/2\), this gives incredibly fast
convergence.

\hypertarget{integer-arithmetic}{%
\subsection{Integer arithmetic}\label{integer-arithmetic}}

While in theory, the Newton iterates should converge for the stableswap
curve, the actual guarantee is nixed by the EVM constraint that it must
all be done in integer arithmetic. In particular, note that when
computing the iterate, we do a big division which can make the value
overshoot, i.e.~go below, the solution \(c\). So the iterates do not
have to be monotonic, and in fact, practical experiments show this can
happen when the pool is severely imbalanced. In such a case, the next
iterate will then go \emph{above} \(c\) and can (as shown in practical
experiments) cause an infinite cycle going below and above the solution.

One very simple way ot stopping such infinite cycles is to check if the
iterate is larger than the previous iterate. If so, we can expect that
iteration has ``bounced'' due to the integer truncation and use the
previous iterate as our return value. This is in fact a standard way to
handle integer arithmetical issues in such monotonically converging uses
of Newton's method.

The stopping criterion used in current stableswap conditions is to check
that the current iterate and previous iterate differs by at most 1. This
does not in fact block infinite cycles. In practice however, such
infinite cycles do not appear to happen under very imbalanced
conditions. Nonetheless the ``bounce'' criterion may be preferred as it
can evidently handle a much greater range of imbalance.

\hypertarget{the-swap-equation}{%
\section{The swap equation}\label{the-swap-equation}}

The stableswap equation allows you to solve for any coin balance given
the other balances and the value of \(D\). This is a fundamental
property needed for enabling swap functionality. Since this is not
derived in the whitepaper, we go through it here.

The stableswap equation can be re-written in the form:

\[ An\left(x_j + \sum_{k\neq j} x_k\right) + D = AnD + \frac{D^{n+1}}{n^n x_j \prod_{k\neq j} x_k} \]

where \(j\) is the out-token index.

Let's denote \(\sum_{k\neq j} x_k\) by \(S'\) and
\(\prod_{k\neq j} x_k\) by \(P'\).

Then we have, after some re-arranging

\[ x_j + S' + \frac{D}{An} = D + \frac{D^{n+1}}{An^{n+1} x_j P'} \]

This becomes

\[ x_j^2 + \left(S' + \frac{D}{An} - D\right) x_j = \frac{D^{n+1}}{An^{n+1}P'}\]

or

\[ x_j^2 + bx_j = c\]

where \(b = S' + \frac{D}{An} - D\) and
\(c = \frac{D^{n+1}}{An^{n+1}P'}\).

This quadratic equation can be solved by Newton's method:

\[ \begin{aligned}
x_j &:= x_j - \frac{x_j^2 + bx_j - c}{2x_j + b}\\
&:= \frac{x_j^2 + c}{2x_j + b} \\
\end{aligned} \]

Note the actual vyper code cleverly defines \(b\) as our \(b\) without
the \(-D\) term. This allows \(b\) to be defined as a
\passthrough{\lstinline!uint256!} since otherwise it could be negative
(although of course \(2x_j + b\) is always positive).

The vyper code should be understandable now:

\begin{lstlisting}[numbers=left, firstnumber=356, label=get_y]
@view
@internal
def get_y(i: int128, j: int128, x: uint256, xp_: uint256[N_COINS]) -> uint256:
    # x in the input is converted to the same price/precision

    assert i != j       # dev: same coin
    assert j >= 0       # dev: j below zero
    assert j < N_COINS  # dev: j above N_COINS

    # should be unreachable, but good for safety
    assert i >= 0
    assert i < N_COINS

    amp: uint256 = self._A()
    D: uint256 = self.get_D(xp_, amp)
    c: uint256 = D
    S_: uint256 = 0
    Ann: uint256 = amp * N_COINS

    _x: uint256 = 0
    for _i in range(N_COINS):
        if _i == i:
            _x = x
        elif _i != j:
            _x = xp_[_i]
        else:
            continue
        S_ += _x
        c = c * D / (_x * N_COINS)
    c = c * D / (Ann * N_COINS)
    b: uint256 = S_ + D / Ann  # - D
    y_prev: uint256 = 0
    y: uint256 = D
    for _i in range(255):
        y_prev = y
        y = (y*y + c) / (2 * y + b - D)            /*@\llabel{quadratic_iteration}@*/
        # Equality with the precision of 1
        if y > y_prev:
            if y - y_prev <= 1:
                break
        else:
            if y_prev - y <= 1:
                break
    return y
\end{lstlisting}

\ref{quadratic_iteration}

So given all the normalized balances (the out-token balance doesn't
matter), we can compute the balance of the out-token that satisfies the
stableswap equation for the given \(D\) and other balances.

This is what's done in the \passthrough{\lstinline!get\_dy!} function in
the stableswap contract:

\begin{lstlisting}[numbers=left, firstnumber=356, label=get_dy]
@view
@external
def get_dy(i: int128, j: int128, dx: uint256) -> uint256:
    # dx and dy in c-units
    rates: uint256[N_COINS] = RATES
    xp: uint256[N_COINS] = self._xp()

    x: uint256 = xp[i] + (dx * rates[i] / PRECISION)
    y: uint256 = self.get_y(i, j, x, xp)
    dy: uint256 = (xp[j] - y - 1) * PRECISION / rates[j]
    _fee: uint256 = self.fee * dy / FEE_DENOMINATOR
    return dy - _fee
\end{lstlisting}

The key logic is given in the lines:

\begin{lstlisting}[numbers=left, firstnumber=364]
y: uint256 = self.get_y(i, j, x, xp)
dy: uint256 = (xp[j] - y - 1) * PRECISION / rates[j]
\end{lstlisting}

As usual the \passthrough{\lstinline!xp!} balances are the virtual
balances, the token balances normalized to be in the same units as
\passthrough{\lstinline!D!} with any rate adjustment to compensate for
changes in value, e.g.~accrued interest.

So by using \passthrough{\lstinline!get\_y!} on the in-token balance
increased by the swap amount \passthrough{\lstinline!dx!}, we can get
the new out-token balance and subtract from the old out-token balances,
which gives us \passthrough{\lstinline!dy!}. This then gets adjusted to
native token units with the fee taken out.

The \passthrough{\lstinline!get\_dy!} isn't actually what's used to do
the exchange, but the \passthrough{\lstinline!exchange!} function does
the identical logic while handling token transfers and other fee logic,
particularly sweeping ``admin fees'', which are the fees going to the
DAO. In any case, the amount \passthrough{\lstinline!dy!} is the same.

Note that an extra ``wei'' is subtracted in
\passthrough{\lstinline!xp[j] - y - 1!}. This is because
\passthrough{\lstinline!y!} might be truncated due to the integer
division, effectively rounding down. So we compensate by increasing
\passthrough{\lstinline!y!} by 1. Defending against value leakage like
this protects the pool from attackers potentially draining the pool,
although it may very slightly penalize the user in some cases.

\hypertarget{fees}{%
\section{Fees}\label{fees}}

Fees enter into the picture in three different ways.

\begin{enumerate}
\def\labelenumi{\arabic{enumi})}
\item
  During an \textbf{exchange}, the out-token amount received by the user
  is reduced. This is added back to the pool, which increases the
  stableswap invariant (the invariant increases when a coin balance
  increases, as can be checked using the usual calculus). This increases
  the balances for LPs, effectively ``auto-compounding'' over time as
  LPs add or remove liquidity.
\item
  When an LP \textbf{adds liquidity} fees are deducted for coin amounts
  that differ from the ``ideal'' balance (same proportions as the pool).
  The reduced input amounts are then used to mint LP tokens.
\item
  When an LP \textbf{removes liquidity imbalanced (or in one coin)} fees
  are yet again deducted for coin amounts differing from an ``ideal''
  withdrawal (same proportions as the pool). The fees here and in adding
  liquidity are designed to total a normal swap fee under some
  assumptions we will spell out below.
\end{enumerate}

\hypertarget{exchange}{%
\subsection{Exchange}\label{exchange}}

\hypertarget{adding-and-removing-liquidity}{%
\subsection{Adding and removing
liquidity}\label{adding-and-removing-liquidity}}

It is important to include a fee when adding or removing liquidity as
otherwise, users are incentived to avoid swaps and merely add liquidity
in one coin and remove in a different coin. While the addition and
removal of liquidity is more gas-intensive than a swap, the gas cost is
fixed, meaning that for large swaps, there is a potentially significant
savings to use this route.

First it should be noted that adding liquidity with coin amounts in the
same proportions as the pool's reserves does not result in fees. The
same goes for the standard \passthrough{\lstinline!remove\_liquidity!},
which withdraws amounts proportional to pool reserves. This avoids
penalizing liquidity provisioning.

The key starting observation is that it is straighforward to increase
\(D\) by any percentage by picking appropriate, proportional deposit
amounts. For example, if you want to increase \(D\) by 2\%, you can do
so increasing each coin balance by 2\%. This follows simply from the
stableswap equation.

Thus when adding amounts the new \(D\) is calculated from the increased
pool reserves. Then the ideal amounts are calculated as the same
percentage of each coin balance as the percentage increase in the new
\(D\).

The fee \(f_{\operatorname{add}}\) is taken from each absolute
difference of a coin deposit amount from the corresponding ideal
balance:

\[ \text{fee deducted} = f_{\operatorname{add}} \cdot |\operatorname{amount}_i - \operatorname{ideal}_i| \]

for each \(i\)-ith coin.

Note the same logic applies for \(f_{\operatorname{remove}}\) (from now
on, we'll assume we use the same fee for adding and removing).

The question then becomes, what value should we pick for
\(f_{\operatorname{add}}\) so the total fees deducted from adding and
removing liquidity equals \(f_{\operatorname{swap}}\)?

In order to simplify this calculation, we make some reasonable
assumptions. First we assume that the amounts involved are very small
compared to the pool reserves. We also suppose the pool has negligible
imbalance.

If we deposit \(a\) in one coin, the increase in \(D\) is by \(a\) (we
can assume \(D\) is the sum of balances and stays so, as it is
balanced). The ideal amounts then become
\(\frac{a}{n}, \dots, \frac{a}{n}\).

The fee deducted on adding liquidity is then:

\[ = f_{\operatorname{add}}\left( \left|a - \frac{a}{n}\right| + \left|0 - \frac{a}{n}\right| + ... + \left|0 - \frac{a}{n}\right| \right) \]

The fees deducted on removing liquidity is also the same, since the
output amount is still \(a\) (pool is balanced).

So we must have:

\[ a \cdot f_{\operatorname{swap}} = 2 \cdot f_{\operatorname{add}} \left( \left|a - \frac{a}{n}\right| + \left|0 - \frac{a}{n}\right| + ... + \left|0 - \frac{a}{n}\right| \right)\]

\[  f_{\operatorname{swap}} = 2 \cdot f_{\operatorname{add}} \left( \left|1 - \frac{1}{n}\right| + \left|0 - \frac{1}{n}\right| + ... + \left|0 - \frac{1}{n}\right| \right)\]

\[  f_{\operatorname{swap}} = 2 \cdot f_{\operatorname{add}} \left( 2 \frac{n-1}{n} \right)\]

\[  f_{\operatorname{add}} = \frac{n}{4(n - 1)} \cdot f_{\operatorname{swap}}\]

\hypertarget{useful-formulas}{%
\section{Useful formulas}\label{useful-formulas}}

\hypertarget{price}{%
\subsection{Price}\label{price}}

Use the auxiliary function:

\[ f(x_1, x_2, ..., x_n, D) = A n \sum_i x_i + D - A n D - \frac{D^{n+1}}{n^n \prod_i x_i} \]

When \(f(x_1, ..., x_n, D) = 0\), we have the stableswap equation. We
will be supposing \(D\) fixed in what follows, so you can consider \(f\)
to be a function of just the \(x_i\)'s.

Computing the partials of \(f\), we get:

\[\begin{aligned}
\frac{\partial f}{\partial x_k} &= A n - \frac{\partial}{\partial x_k}\left(\frac{D^{n+1}}{n^n \prod x_i}\right) \\
&= A n -  \frac{\partial}{\partial x_k}\left(\frac{D^{n+1}}{n^n \prod_{i\neq k} x_i} \cdot \frac{1}{x_k}\right)\\
&= A n + \frac{D^{n+1}}{n^n \prod_{i\neq k} x_i} \cdot \frac{1}{x_k^2} \\
&= A n + \frac{D^{n+1}}{n^n \prod_{i} x_i} \cdot \frac{1}{x_k} \\
\end{aligned}\]

When restricting \(f\) to the level set given by
\(f(x_1, ..., x_n) = 0\), we must have
\(\frac{\partial f}{\partial x_i} dx_i + \frac{\partial f}{\partial x_j} dx_j = 0\).

Now the price is

\[\begin{aligned}
- \frac{\partial x_j}{\partial x_i} &= \frac{\frac{\partial f}{\partial x_i}}{\frac{\partial f}{\partial x_j}}\\[1.2ex]
&= \frac{A n + \frac{D^{n+1}}{n^n \prod_{k} x_k} \cdot \frac{1}{x_i}}{A n + \frac{D^{n+1}}{n^n \prod_{k} x_k} \cdot \frac{1}{x_j}} \\[1.2ex]
&= \left( \frac{x_j}{x_i} \right) \frac{\left(A n x_i + \frac{D^{n+1}}{n^n \prod_{k} x_k}\right)}{\left(A n x_j + \frac{D^{n+1}}{n^n \prod_{k} x_k}\right)} \\[1.0ex]
&= \left( \frac{x_j}{x_i} \right) \frac{\left(\frac{A n^{n+1} x_i \prod_k x_k}{D^{n+1}} + 1 \right)}{\left(\frac{A n^{n+1} x_j \prod_k x_k}{D^{n+1}} + 1 \right)} \\[1.0ex]
\end{aligned}\]

Some sanity checks:

\begin{itemize}
\tightlist
\item
  when \(A\) is 0, the price is \(\frac{x_j}{x_i}\), the price for a
  constant product AMM.
\item
  when \(A \rightarrow \infty\), price is 1, the price for a constant
  sum AMM.
\item
  when \(x_j = x_i\), price is 1.
\end{itemize}

\hypertarget{slippage}{%
\subsection{Slippage}\label{slippage}}

\hypertarget{depth}{%
\subsection{Depth}\label{depth}}

\hypertarget{references}{%
\section*{References}\label{references}}
\addcontentsline{toc}{section}{References}

\hypertarget{refs}{}
\begin{CSLReferences}{1}{0}
\leavevmode\vadjust pre{\hypertarget{ref-curvefi-3pool}{}}%
Curve Finance. 2020. {``StableSwap3Pool.vy.''} 2020.
\url{https://github.com/curvefi/curve-contract/blob/bae23a2fc9a0302d174ea1ffce8219a44a92149c/contracts/pools/3pool/StableSwap3Pool.vy}.

\leavevmode\vadjust pre{\hypertarget{ref-egorov2019}{}}%
Egorov, Michael. 2019. {``StableSwap - Efficient Mechanism for
Stablecoin Liquidity.''}
\url{https://classic.curve.fi/files/stableswap-paper.pdf}.

\end{CSLReferences}

\end{document}
