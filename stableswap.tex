% Options for packages loaded elsewhere
\PassOptionsToPackage{unicode}{hyperref}
\PassOptionsToPackage{hyphens}{url}
%
\documentclass[
]{article}
\usepackage{lmodern}
\usepackage{amssymb,amsmath}
\usepackage{ifxetex,ifluatex}
\ifnum 0\ifxetex 1\fi\ifluatex 1\fi=0 % if pdftex
  \usepackage[T1]{fontenc}
  \usepackage[utf8]{inputenc}
  \usepackage{textcomp} % provide euro and other symbols
\else % if luatex or xetex
  \usepackage{unicode-math}
  \defaultfontfeatures{Scale=MatchLowercase}
  \defaultfontfeatures[\rmfamily]{Ligatures=TeX,Scale=1}
\fi
% Use upquote if available, for straight quotes in verbatim environments
\IfFileExists{upquote.sty}{\usepackage{upquote}}{}
\IfFileExists{microtype.sty}{% use microtype if available
  \usepackage[]{microtype}
  \UseMicrotypeSet[protrusion]{basicmath} % disable protrusion for tt fonts
}{}
\makeatletter
\@ifundefined{KOMAClassName}{% if non-KOMA class
  \IfFileExists{parskip.sty}{%
    \usepackage{parskip}
  }{% else
    \setlength{\parindent}{0pt}
    \setlength{\parskip}{6pt plus 2pt minus 1pt}}
}{% if KOMA class
  \KOMAoptions{parskip=half}}
\makeatother
\usepackage{xcolor}
\IfFileExists{xurl.sty}{\usepackage{xurl}}{} % add URL line breaks if available
\IfFileExists{bookmark.sty}{\usepackage{bookmark}}{\usepackage{hyperref}}
\hypersetup{
  hidelinks,
  pdfcreator={LaTeX via pandoc}}
\urlstyle{same} % disable monospaced font for URLs
\setlength{\emergencystretch}{3em} % prevent overfull lines
\providecommand{\tightlist}{%
  \setlength{\itemsep}{0pt}\setlength{\parskip}{0pt}}
\setcounter{secnumdepth}{-\maxdimen} % remove section numbering

\date{}

\begin{document}

\hypertarget{curve-stableswap-from-whitepaper-to-vyper}{%
\section{Curve Stableswap: From Whitepaper To
Vyper}\label{curve-stableswap-from-whitepaper-to-vyper}}

The stableswap invariant was derived by Michael Egorov and promulgated
in the \href{https://curve.fi/files/stableswap-paper.pdf}{whitepaper},
``StableSwap - efficient mechanism for Stablecoin liquidity''. The
whitepaper clearly explained the invariant and its implications for
DeFi; however, there are differences with how it is implemented in
practice, currently across dozens of live contracts across Ethereum and
other layer 2s and chains.

In particular, implementation of fees, both for exchanges and
adding/removing liquidity, is not explained in the whitepaper. Also, the
actual solution procedure for the invariant, particularly in integer
arithmetic, is not given.

The practitioner seeking to understand the live functionality of the
stableswap pools must look toward the vyper code for help, which while
very readable, has minimal comments and explanation (indeed some
comments are even wrong!). To understand the vyper code, the reader must
have a solid grasp of the whitepaper in order to translate to the
appropriate variables and understand various tweaks needed for
implementation.

This note seeks to close the gap between the whitepaper and the vyper
contracts. It seeks to give a consistent derivation of the necessary
mathematics, using the notation and language of the contracts.

\hypertarget{preliminaries-notation-and-conventions}{%
\subsection{Preliminaries (notation and
conventions)}\label{preliminaries-notation-and-conventions}}

\hypertarget{stableswap-equation}{%
\subsubsection{Stableswap equation}\label{stableswap-equation}}

This is the original stableswap equation:
\[ A \cdot n^n  \sum_i x_i + D = A \cdot n^n \cdot D + \frac{D^{n+1}}{n^n \prod_i x_i}\]

In the vyper code, the amplification constant \(A\) actually refers to
\(A \cdot n^{n-1}\), so the equation becomes:

\[ A \cdot n  \sum_i x_i + D = A \cdot n \cdot D + \frac{D^{n+1}}{n^n \prod_i x_i}\]

This is the form we use for all our derivations.

\hypertarget{coin-balances}{%
\subsubsection{Coin balances}\label{coin-balances}}

We denote the coin balances (as in the contracts) with \(x_i\), \(x_j\)
etc. In the context of a swap, \(i\) is the ``in'' index and \(j\) is
the ``out'' index.

The quantities though are in special units. They are \emph{not} native
token units. For example, if \(x_i\) represents the USDC amount, one
whole token amount would not be \(1000000\) as might be assumed from the
6 decimals for USDC. Instead \(x_i\) would be \(1000000000000000000\)
(18 zeroes). All the \(x\) balances should be assumed to be in the same
units as \(D\). For lack of a better term, sometimes we will call these
\emph{virtual} units (as the amount of \(D\) per LP token is often
referred to as ``virtual price'') and we will call the \(x\) balances
\emph{virtual balances}.

While putting balances into virtual units often involves only a change
of decimals, this is not the correct way of thinking about the process.
The stableswap equation assumes a 1:1 peg between coins. This means the
balances being used must reflect the proper value in the common units of
D being used. For the example of USDC, this means adjusting simply the
decimals. For a rebasing token however, it may not be. Indeed, for
metapools, when exchange the basepool's tokens for the main stable, the
basepool token conversion into D units must take into account the
accrued yield. This is done by multiplying the amount by the basepool
virtual price.

\hypertarget{solving-for-d}{%
\subsection{\texorpdfstring{Solving for
\(D\)}{Solving for D}}\label{solving-for-d}}

Since the arithmetic mean is greater than the geometric mean (unless the
balances \(x_i\) are equal, in which case the means are identical), the
form of the equation suggests that there ought to be a \(D\) in-between
the means that satisfies the equation.

To see this rigorously, we use the auxiliary function:

\[ f(D) = A \cdot n \cdot (D - A \cdot n  \sum_i x_i)  + D \cdot (\frac{D^{n}}{n^n \prod_i x_i} - 1) \]

Let \(P = n\ (\prod_i x_i)^{\frac{1}{n}}\) and \(S = \sum_i x_i\). This
is a continuous function (away from zero balances) with \(f(P) < 0\) and
\(f(S) > 0\). So there is a D such that \(P < D < S\) and \(f(D) = 0\).
In particular, note

\[ f'(D) = A\cdot n + (n+1) \frac{D^n}{n^n \prod_i x_i} - 1 \]

the derivative of \(f\), is positive (assuming \(A >= 1\)), so \(f\) is
strictly increasing and there is a unique \(D\) that solves
\(f(D) = 0\).

\hypertarget{newtons-method}{%
\subsubsection{Newton's method}\label{newtons-method}}

The stableswap contracts utilize Newton's method to solve for \(D\). It
is easy to check \(f'' > 0\), i.e.~\(f\) is convex. An elementary
argument shows that this guarantees convergence of Newton's method
starting with initial point \(S\) to the solution.

The vyper code (from 3Pool) is:

\begin{verbatim}
@pure
@internal
def get_D(xp: uint256[N_COINS], amp: uint256) -> uint256:
    S: uint256 = 0
    for _x in xp:
        S += _x
    if S == 0:
        return 0

    Dprev: uint256 = 0
    D: uint256 = S
    Ann: uint256 = amp * N_COINS
    for _i in range(255):
        D_P: uint256 = D
        for _x in xp:
            D_P = D_P * D / (_x * N_COINS)  # If division by 0, this will be borked: only withdrawal will work. And that is good
        Dprev = D
        D = (Ann * S + D_P * N_COINS) * D / ((Ann - 1) * D + (N_COINS + 1) * D_P)
        # Equality with the precision of 1
        if D > Dprev:
            if D - Dprev <= 1:
                break
        else:
            if Dprev - D <= 1:
                break
    return D
\end{verbatim}

This code is used with minimal difference between all the stableswap
contracts. Later versions choose to revert if the 255 iterations are
exhausted before converging.

The iterative formula is easily derived: \[\begin{aligned}
d_{k+1} &= d_k - \frac{f(d_k)}{f'(d_k)} \\
&= d_k - \frac{A n (d_k - A \sum_i x_i)  + d_k(\frac{d_k^{n}}{n^n \prod_i x_i} - 1)}{\frac{(n+1)d_k^n}{n^n \prod_i x_i} + An - 1} \\
&= \frac{An\sum_i x_i + \frac{nd_k^{n+1}}{n^n \prod_i x_i}}{\frac{(n+1)d_k^n}{n^n\prod_i x_i} + An - 1} \\
&= \frac{AnS + nD_p(d_k)}{\frac{D_p(d_k)}{d_k} + An - 1} \\
&= \frac{(AnS + nD_p(d_k))d_k}{D_p(d_k) + (An-1)d_k}\\
\end{aligned}\]

where \(S = \sum_i x_i\) and
\(D_p(d_k) = \frac{d_k^{n+1}}{n^n \prod_i x_i}\)

\hypertarget{quadratic-convergence}{%
\paragraph{Quadratic convergence}\label{quadratic-convergence}}

Convergence is easily argued based on convexity of \(f\). However we
need much better than that, we need at least linear convergence,
otherwise 255 iterations is not sufficient. Also, in practice, exceeding
more than half a dozen iterations is not sufficiently gas efficient
enough to be competitive.

\hypertarget{integer-arithmetic}{%
\subsubsection{Integer arithmetic}\label{integer-arithmetic}}

\hypertarget{the-swap-equation}{%
\subsection{The swap equation}\label{the-swap-equation}}

The stableswap equation allows you to solve for any coin balance given
the other balances and the value of \(D\). This is a fundamental
property needed for enabling swap functionality. Since this is not
derived in the whitepaper, we go through it here.

The stableswap equation can be re-written in the form:

\[ An\left(x_j + \sum_{k\neq j} x_k\right) + D = AnD + \frac{D^{n+1}}{n^n x_j \prod_{k\neq j} x_k} \]

where \(j\) is the out-token index.

Let's denote \(\sum_{k\neq j} x_k\) by \(S'\) and
\(\prod_{k\neq j} x_k\) by \(P'\).

Then we have, after some re-arranging

\[ x_j + S' + \frac{D}{An} = D + \frac{D^{n+1}}{An^{n+1} x_j P'} \]

This becomes

\[ x_j^2 + \left(S' + \frac{D}{An} - D\right) x_j = \frac{D^{n+1}}{An^{n+1}P'}\]

or

\[ x_j^2 + bx_j = c\]

where \(b = S' + \frac{D}{An} - D\) and
\(c = \frac{D^{n+1}}{An^{n+1}P'}\).

This quadratic equation can be solved by Newton's method:

\[ \begin{aligned}
x_j &:= x_j - \frac{x_j^2 + bx_j - c}{2x_j + b}\\
&:= \frac{x_j^2 + c}{2x_j + b} \\
\end{aligned} \]

Note the actual vyper code cleverly defines \(b\) as our \(b\) without
the \(-D\) term. This allows \(b\) to be defined as a \texttt{uint256}
since otherwise it could be negative (although of course \(2x_j + b\) is
always positive).

\hypertarget{fees}{%
\subsection{Fees}\label{fees}}

\hypertarget{exchange}{%
\subsubsection{Exchange}\label{exchange}}

\hypertarget{adding-and-removing-liquidity}{%
\subsubsection{Adding and removing
liquidity}\label{adding-and-removing-liquidity}}

\hypertarget{balanced-deposits-and-withdrawals}{%
\paragraph{Balanced deposits and
withdrawals}\label{balanced-deposits-and-withdrawals}}

\hypertarget{removing-one-coin}{%
\paragraph{Removing one coin}\label{removing-one-coin}}

\hypertarget{useful-formulas}{%
\subsection{Useful formulas}\label{useful-formulas}}

\hypertarget{price}{%
\subsubsection{Price}\label{price}}

\hypertarget{slippage}{%
\subsubsection{Slippage}\label{slippage}}

\hypertarget{depth}{%
\subsubsection{Depth}\label{depth}}

\end{document}
